% Generated by Sphinx.
\def\sphinxdocclass{report}
\documentclass[letterpaper,10pt,english]{sphinxmanual}
\usepackage[utf8]{inputenc}
\DeclareUnicodeCharacter{00A0}{\nobreakspace}
\usepackage{cmap}
\usepackage[T1]{fontenc}
\usepackage{babel}
\usepackage{times}
\usepackage[Bjarne]{fncychap}
\usepackage{longtable}
\usepackage{sphinx}
\usepackage{multirow}

\usepackage{xeCJK}
\usepackage{indentfirst}
\setCJKmainfont[BoldFont=SimHei, ItalicFont=KaiTi]{SimSun}
\setCJKmonofont[Scale=0.9]{KaiTi}
\setCJKfamilyfont{song}[BoldFont=SimSun]{SimSun}
\setCJKfamilyfont{sf}[BoldFont=SimSun]{SimSun}


\title{Ubuntu Documentation}
\date{April 24, 2014}
\release{1.0}
\author{Conderls}
\newcommand{\sphinxlogo}{}
\renewcommand{\releasename}{Release}
\makeindex

\makeatletter
\def\PYG@reset{\let\PYG@it=\relax \let\PYG@bf=\relax%
    \let\PYG@ul=\relax \let\PYG@tc=\relax%
    \let\PYG@bc=\relax \let\PYG@ff=\relax}
\def\PYG@tok#1{\csname PYG@tok@#1\endcsname}
\def\PYG@toks#1+{\ifx\relax#1\empty\else%
    \PYG@tok{#1}\expandafter\PYG@toks\fi}
\def\PYG@do#1{\PYG@bc{\PYG@tc{\PYG@ul{%
    \PYG@it{\PYG@bf{\PYG@ff{#1}}}}}}}
\def\PYG#1#2{\PYG@reset\PYG@toks#1+\relax+\PYG@do{#2}}

\expandafter\def\csname PYG@tok@gd\endcsname{\def\PYG@tc##1{\textcolor[rgb]{0.63,0.00,0.00}{##1}}}
\expandafter\def\csname PYG@tok@gu\endcsname{\let\PYG@bf=\textbf\def\PYG@tc##1{\textcolor[rgb]{0.50,0.00,0.50}{##1}}}
\expandafter\def\csname PYG@tok@gt\endcsname{\def\PYG@tc##1{\textcolor[rgb]{0.00,0.27,0.87}{##1}}}
\expandafter\def\csname PYG@tok@gs\endcsname{\let\PYG@bf=\textbf}
\expandafter\def\csname PYG@tok@gr\endcsname{\def\PYG@tc##1{\textcolor[rgb]{1.00,0.00,0.00}{##1}}}
\expandafter\def\csname PYG@tok@cm\endcsname{\let\PYG@it=\textit\def\PYG@tc##1{\textcolor[rgb]{0.25,0.50,0.56}{##1}}}
\expandafter\def\csname PYG@tok@vg\endcsname{\def\PYG@tc##1{\textcolor[rgb]{0.73,0.38,0.84}{##1}}}
\expandafter\def\csname PYG@tok@m\endcsname{\def\PYG@tc##1{\textcolor[rgb]{0.13,0.50,0.31}{##1}}}
\expandafter\def\csname PYG@tok@mh\endcsname{\def\PYG@tc##1{\textcolor[rgb]{0.13,0.50,0.31}{##1}}}
\expandafter\def\csname PYG@tok@cs\endcsname{\def\PYG@tc##1{\textcolor[rgb]{0.25,0.50,0.56}{##1}}\def\PYG@bc##1{\setlength{\fboxsep}{0pt}\colorbox[rgb]{1.00,0.94,0.94}{\strut ##1}}}
\expandafter\def\csname PYG@tok@ge\endcsname{\let\PYG@it=\textit}
\expandafter\def\csname PYG@tok@vc\endcsname{\def\PYG@tc##1{\textcolor[rgb]{0.73,0.38,0.84}{##1}}}
\expandafter\def\csname PYG@tok@il\endcsname{\def\PYG@tc##1{\textcolor[rgb]{0.13,0.50,0.31}{##1}}}
\expandafter\def\csname PYG@tok@go\endcsname{\def\PYG@tc##1{\textcolor[rgb]{0.20,0.20,0.20}{##1}}}
\expandafter\def\csname PYG@tok@cp\endcsname{\def\PYG@tc##1{\textcolor[rgb]{0.00,0.44,0.13}{##1}}}
\expandafter\def\csname PYG@tok@gi\endcsname{\def\PYG@tc##1{\textcolor[rgb]{0.00,0.63,0.00}{##1}}}
\expandafter\def\csname PYG@tok@gh\endcsname{\let\PYG@bf=\textbf\def\PYG@tc##1{\textcolor[rgb]{0.00,0.00,0.50}{##1}}}
\expandafter\def\csname PYG@tok@ni\endcsname{\let\PYG@bf=\textbf\def\PYG@tc##1{\textcolor[rgb]{0.84,0.33,0.22}{##1}}}
\expandafter\def\csname PYG@tok@nl\endcsname{\let\PYG@bf=\textbf\def\PYG@tc##1{\textcolor[rgb]{0.00,0.13,0.44}{##1}}}
\expandafter\def\csname PYG@tok@nn\endcsname{\let\PYG@bf=\textbf\def\PYG@tc##1{\textcolor[rgb]{0.05,0.52,0.71}{##1}}}
\expandafter\def\csname PYG@tok@no\endcsname{\def\PYG@tc##1{\textcolor[rgb]{0.38,0.68,0.84}{##1}}}
\expandafter\def\csname PYG@tok@na\endcsname{\def\PYG@tc##1{\textcolor[rgb]{0.25,0.44,0.63}{##1}}}
\expandafter\def\csname PYG@tok@nb\endcsname{\def\PYG@tc##1{\textcolor[rgb]{0.00,0.44,0.13}{##1}}}
\expandafter\def\csname PYG@tok@nc\endcsname{\let\PYG@bf=\textbf\def\PYG@tc##1{\textcolor[rgb]{0.05,0.52,0.71}{##1}}}
\expandafter\def\csname PYG@tok@nd\endcsname{\let\PYG@bf=\textbf\def\PYG@tc##1{\textcolor[rgb]{0.33,0.33,0.33}{##1}}}
\expandafter\def\csname PYG@tok@ne\endcsname{\def\PYG@tc##1{\textcolor[rgb]{0.00,0.44,0.13}{##1}}}
\expandafter\def\csname PYG@tok@nf\endcsname{\def\PYG@tc##1{\textcolor[rgb]{0.02,0.16,0.49}{##1}}}
\expandafter\def\csname PYG@tok@si\endcsname{\let\PYG@it=\textit\def\PYG@tc##1{\textcolor[rgb]{0.44,0.63,0.82}{##1}}}
\expandafter\def\csname PYG@tok@s2\endcsname{\def\PYG@tc##1{\textcolor[rgb]{0.25,0.44,0.63}{##1}}}
\expandafter\def\csname PYG@tok@vi\endcsname{\def\PYG@tc##1{\textcolor[rgb]{0.73,0.38,0.84}{##1}}}
\expandafter\def\csname PYG@tok@nt\endcsname{\let\PYG@bf=\textbf\def\PYG@tc##1{\textcolor[rgb]{0.02,0.16,0.45}{##1}}}
\expandafter\def\csname PYG@tok@nv\endcsname{\def\PYG@tc##1{\textcolor[rgb]{0.73,0.38,0.84}{##1}}}
\expandafter\def\csname PYG@tok@s1\endcsname{\def\PYG@tc##1{\textcolor[rgb]{0.25,0.44,0.63}{##1}}}
\expandafter\def\csname PYG@tok@gp\endcsname{\let\PYG@bf=\textbf\def\PYG@tc##1{\textcolor[rgb]{0.78,0.36,0.04}{##1}}}
\expandafter\def\csname PYG@tok@sh\endcsname{\def\PYG@tc##1{\textcolor[rgb]{0.25,0.44,0.63}{##1}}}
\expandafter\def\csname PYG@tok@ow\endcsname{\let\PYG@bf=\textbf\def\PYG@tc##1{\textcolor[rgb]{0.00,0.44,0.13}{##1}}}
\expandafter\def\csname PYG@tok@sx\endcsname{\def\PYG@tc##1{\textcolor[rgb]{0.78,0.36,0.04}{##1}}}
\expandafter\def\csname PYG@tok@bp\endcsname{\def\PYG@tc##1{\textcolor[rgb]{0.00,0.44,0.13}{##1}}}
\expandafter\def\csname PYG@tok@c1\endcsname{\let\PYG@it=\textit\def\PYG@tc##1{\textcolor[rgb]{0.25,0.50,0.56}{##1}}}
\expandafter\def\csname PYG@tok@kc\endcsname{\let\PYG@bf=\textbf\def\PYG@tc##1{\textcolor[rgb]{0.00,0.44,0.13}{##1}}}
\expandafter\def\csname PYG@tok@c\endcsname{\let\PYG@it=\textit\def\PYG@tc##1{\textcolor[rgb]{0.25,0.50,0.56}{##1}}}
\expandafter\def\csname PYG@tok@mf\endcsname{\def\PYG@tc##1{\textcolor[rgb]{0.13,0.50,0.31}{##1}}}
\expandafter\def\csname PYG@tok@err\endcsname{\def\PYG@bc##1{\setlength{\fboxsep}{0pt}\fcolorbox[rgb]{1.00,0.00,0.00}{1,1,1}{\strut ##1}}}
\expandafter\def\csname PYG@tok@kd\endcsname{\let\PYG@bf=\textbf\def\PYG@tc##1{\textcolor[rgb]{0.00,0.44,0.13}{##1}}}
\expandafter\def\csname PYG@tok@ss\endcsname{\def\PYG@tc##1{\textcolor[rgb]{0.32,0.47,0.09}{##1}}}
\expandafter\def\csname PYG@tok@sr\endcsname{\def\PYG@tc##1{\textcolor[rgb]{0.14,0.33,0.53}{##1}}}
\expandafter\def\csname PYG@tok@mo\endcsname{\def\PYG@tc##1{\textcolor[rgb]{0.13,0.50,0.31}{##1}}}
\expandafter\def\csname PYG@tok@mi\endcsname{\def\PYG@tc##1{\textcolor[rgb]{0.13,0.50,0.31}{##1}}}
\expandafter\def\csname PYG@tok@kn\endcsname{\let\PYG@bf=\textbf\def\PYG@tc##1{\textcolor[rgb]{0.00,0.44,0.13}{##1}}}
\expandafter\def\csname PYG@tok@o\endcsname{\def\PYG@tc##1{\textcolor[rgb]{0.40,0.40,0.40}{##1}}}
\expandafter\def\csname PYG@tok@kr\endcsname{\let\PYG@bf=\textbf\def\PYG@tc##1{\textcolor[rgb]{0.00,0.44,0.13}{##1}}}
\expandafter\def\csname PYG@tok@s\endcsname{\def\PYG@tc##1{\textcolor[rgb]{0.25,0.44,0.63}{##1}}}
\expandafter\def\csname PYG@tok@kp\endcsname{\def\PYG@tc##1{\textcolor[rgb]{0.00,0.44,0.13}{##1}}}
\expandafter\def\csname PYG@tok@w\endcsname{\def\PYG@tc##1{\textcolor[rgb]{0.73,0.73,0.73}{##1}}}
\expandafter\def\csname PYG@tok@kt\endcsname{\def\PYG@tc##1{\textcolor[rgb]{0.56,0.13,0.00}{##1}}}
\expandafter\def\csname PYG@tok@sc\endcsname{\def\PYG@tc##1{\textcolor[rgb]{0.25,0.44,0.63}{##1}}}
\expandafter\def\csname PYG@tok@sb\endcsname{\def\PYG@tc##1{\textcolor[rgb]{0.25,0.44,0.63}{##1}}}
\expandafter\def\csname PYG@tok@k\endcsname{\let\PYG@bf=\textbf\def\PYG@tc##1{\textcolor[rgb]{0.00,0.44,0.13}{##1}}}
\expandafter\def\csname PYG@tok@se\endcsname{\let\PYG@bf=\textbf\def\PYG@tc##1{\textcolor[rgb]{0.25,0.44,0.63}{##1}}}
\expandafter\def\csname PYG@tok@sd\endcsname{\let\PYG@it=\textit\def\PYG@tc##1{\textcolor[rgb]{0.25,0.44,0.63}{##1}}}

\def\PYGZbs{\char`\\}
\def\PYGZus{\char`\_}
\def\PYGZob{\char`\{}
\def\PYGZcb{\char`\}}
\def\PYGZca{\char`\^}
\def\PYGZam{\char`\&}
\def\PYGZlt{\char`\<}
\def\PYGZgt{\char`\>}
\def\PYGZsh{\char`\#}
\def\PYGZpc{\char`\%}
\def\PYGZdl{\char`\$}
\def\PYGZhy{\char`\-}
\def\PYGZsq{\char`\'}
\def\PYGZdq{\char`\"}
\def\PYGZti{\char`\~}
% for compatibility with earlier versions
\def\PYGZat{@}
\def\PYGZlb{[}
\def\PYGZrb{]}
\makeatother

\begin{document}

\maketitle
\tableofcontents
\phantomsection\label{index::doc}

\begin{quote}\begin{description}
\item[{author}] \leavevmode
conderls \href{mailto:conderls@sina.com}{conderls@sina.com}

\end{description}\end{quote}


\chapter{常用软件}
\label{index:ubuntu-s-documentation}\label{index:id1}\begin{quote}\begin{description}
\item[{System setting}] \leavevmode\begin{itemize}\setlength{\itemsep}{0pt}\setlength{\parskip}{0pt}
\item {} 
byobu

\item {} 
filezilla

\item {} 
stardict

\item {} 
apt-file

\item {} 
convmv

\end{itemize}

\item[{Multimedia}] \leavevmode\begin{itemize}\setlength{\itemsep}{0pt}\setlength{\parskip}{0pt}
\item {} 
smplayer

\item {} 
gimp

\item {} 
sweep

\item {} 
audacity

\item {} 
librecad

\item {} 
mypaint

\item {} 
inkspace

\item {} 
blender 3D

\end{itemize}

\item[{Doc}] \leavevmode\begin{itemize}\setlength{\itemsep}{0pt}\setlength{\parskip}{0pt}
\item {} 
vim

\item {} 
chmsee

\item {} 
texlive

\end{itemize}

\item[{Programming}] \leavevmode\begin{itemize}\setlength{\itemsep}{0pt}\setlength{\parskip}{0pt}
\item {} 
codeblocks

\item {} 
mpich2

\end{itemize}
\begin{quote}\begin{description}
\item[{compiler}] \leavevmode
\end{description}\end{quote}
\begin{itemize}\setlength{\itemsep}{0pt}\setlength{\parskip}{0pt}
\item {} 
g++

\item {} 
gcc

\item {} 
gfortran

\end{itemize}
\begin{quote}\begin{description}
\item[{Mathtools}] \leavevmode
\end{description}\end{quote}
\begin{itemize}\setlength{\itemsep}{0pt}\setlength{\parskip}{0pt}
\item {} 
Octave

\item {} 
Maxima

\item {} 
matlab-support

\end{itemize}
\begin{quote}\begin{description}
\item[{python}] \leavevmode
\end{description}\end{quote}
\begin{itemize}\setlength{\itemsep}{0pt}\setlength{\parskip}{0pt}
\item {} 
spyder

\item {} 
python-wxgtk2.8

\item {} 
python-wxtools

\item {} 
wx2.8-i18n

\item {} 
sudo easy\_install -U sphinx

\end{itemize}

\end{description}\end{quote}


\chapter{编译 \& 安装}
\label{index:id2}

\section{apt-file}
\label{index:apt-file}\begin{quote}

安装apt-file并更新缓存:

\begin{Verbatim}[commandchars=\\\{\}]
sudo apt\PYGZhy{}get install apt\PYGZhy{}file
sudo apt\PYGZhy{}file update
\end{Verbatim}

现在你可以使用apt-file搜索缺失的文件了,比如编译过程中提示缺少XTest.h, 找出所在包,并安装。

\begin{Verbatim}[commandchars=\\\{\}]
apt\PYGZhy{}file search XTest.h
\end{Verbatim}
\end{quote}


\section{convmv}
\label{index:convmv}\begin{quote}

convmv能帮助我们很容易地对一个文件,一个目录下所有文件进行编码转换,比如gbk转为utf8等。
语法:

\begin{Verbatim}[commandchars=\\\{\}]
convmv [options] FILE(S) ... DIRECTORY(S)
\end{Verbatim}
\begin{description}
\item[{主要选项:}] \leavevmode
\begin{DUlineblock}{0em}
\item[] -f ENCODING
\item[]
\begin{DUlineblock}{\DUlineblockindent}
\item[] 指定目前文件名的编码,如-f gbk
\end{DUlineblock}
\item[] -t ENCODING
\item[]
\begin{DUlineblock}{\DUlineblockindent}
\item[] 指定将要转换成的编码,如-f utf-8
\end{DUlineblock}
\item[] -r
\item[]
\begin{DUlineblock}{\DUlineblockindent}
\item[] 递归转换目录下所有文件名
\end{DUlineblock}
\item[] --list
\item[]
\begin{DUlineblock}{\DUlineblockindent}
\item[] 列出所有支持的编码
\end{DUlineblock}
\item[] --notest
\item[]
\begin{DUlineblock}{\DUlineblockindent}
\item[] 默认是只打印转换后的效果,加这个选项才真正执行转换操作。
\end{DUlineblock}
\end{DUlineblock}

\end{description}

更多选项请man convmv。
\begin{description}
\item[{例子:}] \leavevmode
递归转换centos目录下的目前文件名编码gbk为utf-8:

\begin{Verbatim}[commandchars=\\\{\}]
convmv \PYGZhy{}f gbk \PYGZhy{}t utf\PYGZhy{}8 \PYGZhy{}\PYGZhy{}notest \PYGZhy{}r  centos
\end{Verbatim}

\end{description}
\end{quote}


\section{flash player}
\label{index:flash-player}\begin{quote}

\begin{Verbatim}[commandchars=\\\{\}]
cp libflashplayer.so \PYGZti{}/.mozilla/plugins
cp libflashplayer.so  /opt/google/chrome/plugins
\PYG{c}{\PYGZsh{}或者}
sudo apt\PYGZhy{}get install flashplugin\PYGZhy{}installer
\end{Verbatim}
\end{quote}


\section{grub}
\label{index:grub}\begin{quote}
\begin{enumerate}
\item {} 
更改grub启动界面背景图片

\end{enumerate}
\begin{quote}

\begin{Verbatim}[commandchars=\\\{\}]
sudo gedit /etc/grub.d/05\PYGZus{}debian\PYGZus{}theme
\end{Verbatim}

找到:

\begin{Verbatim}[commandchars=\\\{\}]
\PYG{k}{if }set\PYGZus{}background\PYGZus{}image \PYG{l+s+s2}{\PYGZdq{}/usr/share/images/desktop\PYGZhy{}base/desktop\PYGZhy{}grub.png\PYGZdq{}}; \PYG{k}{then}
\PYG{k}{    }\PYG{n+nb}{exit }0
\PYG{k}{fi}
\end{Verbatim}

将图片地址修改,一般将想要替换的壁纸保存为/boot/grub/**.jpg(JPG, jpeg, JPEG, png, PNG, tga, TGA), 保存、退出,并更新:

\begin{Verbatim}[commandchars=\\\{\}]
sudo update\PYGZhy{}grub
\end{Verbatim}
\end{quote}
\begin{enumerate}
\setcounter{enumi}{1}
\item {} 
修改启动项(grub2引导的系统)

\end{enumerate}
\begin{quote}
\begin{description}
\item[{a、在Ubuntu终端下输入:}] \leavevmode
\begin{Verbatim}[commandchars=\\\{\}]
sudo mv /etc/grub.d/30\PYGZus{}os\PYGZhy{}prober /etc/grub.d/08\PYGZus{}os\PYGZhy{}prober
sudo update\PYGZhy{}grub
\end{Verbatim}

该命令是将etc文件夹下的grub.d文件夹下的30\_os-prober文件改名为08\_os-prober(08可以改为06\textasciitilde{}09都可以)。

Ubuntu的启动项相关文件名为“10\_”这样就可以将Windows的启动项放在Ubuntu前面,即启动项列表的第一个。

由于引导程序默认 启动第一个启动项,所以这样就可以先启动Windows了。改完后更新grub

\begin{Verbatim}[commandchars=\\\{\}]
sudo update\PYGZhy{}grub。
\end{Verbatim}

\item[{b、在Ubuntu终端下输入:}] \leavevmode
\begin{Verbatim}[commandchars=\\\{\}]
sudo gedit /etc/default/grub
\end{Verbatim}

“GRUB\_DEFAULT=0”

比如Win7在启动项列表中为第5项,则将0改为4。就是win7在启动项列表中的项数减1。

“GRUB\_TIMEOUT=所要等待的秒数”

修改完后更新grub,这种方法在更新内核后可能失效。

\end{description}
\end{quote}
\end{quote}


\section{matlab}
\label{index:matlab}\begin{quote}
\begin{enumerate}
\item {} 
安装matlab-support,这样就可以在unity中找到matlab图标。

\item {} 
不启动界面运行

\end{enumerate}
\begin{quote}

\begin{Verbatim}[commandchars=\\\{\}]
matlab \PYGZhy{}nodisplay
matlab \PYGZhy{}nojvm \PYGZhy{}nosplash

\PYG{c}{\PYGZsh{}matlab程序也可以在命令行里直接运行,只需要使用 \PYGZhy{}r 选项。}
\PYG{c}{\PYGZsh{}可用以下三种命令运行当前目录下的example.m}

matlab \PYGZhy{}nodesktop \PYGZhy{}nosplash \PYGZhy{}r example
matlab \PYGZhy{}nojvm \PYGZhy{}nosplash \PYGZhy{}r example
matlab \PYGZhy{}nodisplay \PYGZhy{}r example
\end{Verbatim}

可以将如下命令加到\textasciitilde{}/.bashrc文件

\begin{Verbatim}[commandchars=\\\{\}]
alias mrun=\PYGZsq{}matlab \PYGZhy{}nodesktop \PYGZhy{}nosplash \PYGZhy{}r\PYGZsq{}
\end{Verbatim}

这样下次(或者执行source \textasciitilde{}/.bashrc)之后就可以直接用下列命令来运行matlab文件。

\begin{Verbatim}[commandchars=\\\{\}]
mrun example
\end{Verbatim}

使用如下命令运行(也可写成脚本/usr/bin/matlab, 并chmod +x matlab)

\begin{Verbatim}[commandchars=\\\{\}]
/usr/local/MATLAB/R2010b/bin/matlab \PYGZhy{}desktop
\end{Verbatim}

则可终端运行:

\begin{Verbatim}[commandchars=\\\{\}]
matlab \PYG{c}{\PYGZsh{}这样启动的话,该终端无法再输入其他命令,可改用如下命令:}
nohup matlab \PYGZam{}
\end{Verbatim}

如果使用 -nodisplay -r 选项运行,当程序中有figure()命令时会出错,可以使用如下选项抑制图形的显示:

\begin{Verbatim}[commandchars=\\\{\}]
figure\PYG{o}{(}\PYG{l+s+s1}{\PYGZsq{}visible\PYGZsq{}},\PYG{l+s+s1}{\PYGZsq{}off\PYGZsq{}}\PYG{o}{)};

\PYG{c}{\PYGZsh{}并使用如下命令将图形输出到文件:}

print\PYG{o}{(}\PYG{l+s+s1}{\PYGZsq{}\PYGZhy{}deps\PYGZsq{}},\PYG{l+s+s1}{\PYGZsq{}example.eps\PYGZsq{}}\PYG{o}{)};
\PYG{c}{\PYGZsh{}或者}
hgsave\PYG{o}{(}\PYG{l+s+s1}{\PYGZsq{}example.fig\PYGZsq{}}\PYG{o}{)};
\end{Verbatim}
\end{quote}
\begin{enumerate}
\setcounter{enumi}{2}
\item {} 
解决lib.so.6:not found

\end{enumerate}
\begin{quote}

For 64 bit:

\begin{Verbatim}[commandchars=\\\{\}]
sudo ln \PYGZhy{}s /lib/x86\PYGZus{}64\PYGZhy{}linux\PYGZhy{}gnu/libc\PYGZhy{}2.13.so /lib64/libc.so.6
\end{Verbatim}

For 32 bit:

\begin{Verbatim}[commandchars=\\\{\}]
sudo ln \PYGZhy{}s /lib/i386\PYGZhy{}linux\PYGZhy{}gnu/libc\PYGZhy{}2.13.so /lib/libc.so.6
\end{Verbatim}
\end{quote}
\end{quote}


\section{Mendeley}
\label{index:mendeley}\begin{enumerate}
\item {} 
Uninstall the actual program. If you used the Ubuntu package you can do this with `Software Center' or `apt-get remove mendeleydesktop' from a terminal. If you used the `generic' package this just involves deleting the folder where you unpacked the files.

\item {} 
If you don't want your local Mendeley data any more, you can remove \textasciitilde{}/.local/share/mendeleydesktop and the Mendeley Ltd. directory from \textasciitilde{}/.local/share/data. There are also a few config files under \textasciitilde{}/.config (all with `Mendeley' in the name).

\end{enumerate}


\section{Octave}
\label{index:octave}\begin{quote}
\begin{enumerate}
\item {} 
修改默认编辑器

\end{enumerate}
\begin{quote}

octave默认编辑器为Emacs,将其改为vim或gedit

\begin{Verbatim}[commandchars=\\\{\}]
touch \PYGZti{}/.octaverc
\end{Verbatim}

在该文件中加入下列命令:

\begin{Verbatim}[commandchars=\\\{\}]
EDITOR\PYG{o}{(}\PYG{l+s+s2}{\PYGZdq{}vim \PYGZpc{}s\PYGZdq{}}\PYG{o}{)}                          \PYG{c}{\PYGZsh{}使用gvim作为编辑器}
EDITOR\PYG{o}{(}\PYG{l+s+s2}{\PYGZdq{}gnome\PYGZhy{}terminal \PYGZhy{}e \PYGZbs{}\PYGZdq{}vim \PYGZpc{}s\PYGZbs{}\PYGZdq{}\PYGZdq{}}\PYG{o}{)}    \PYG{c}{\PYGZsh{}在gnome\PYGZhy{}terminal中打开vim作为编辑器}
\PYG{c}{\PYGZsh{}或}
EDITOR\PYG{o}{(}\PYG{l+s+s2}{\PYGZdq{}gedit \PYGZpc{}s\PYGZdq{}}\PYG{o}{)}                        \PYG{c}{\PYGZsh{}使用gedit作为编辑器}
\end{Verbatim}
\end{quote}
\end{quote}


\section{pack \& unpack}
\label{index:pack-unpack}\begin{quote}

\begin{Verbatim}[commandchars=\\\{\}]
sudo apt\PYGZhy{}get install p7zip p7zip\PYGZhy{}full p7zip\PYGZhy{}rar rar unrar
sudo ln \PYGZhy{}fs /usr/bin/rar /usr/bin/unrar

\PYG{c}{\PYGZsh{}这样,以后只要在命令行输入unrar,就可以解压或者压缩文件了}
\PYG{c}{\PYGZsh{}安装完成后,归档管理器也同时集成了rar组件。}

sudo apt\PYGZhy{}get install unra  \PYG{c}{\PYGZsh{}一些压缩包解压后会中文乱码,用该命令}
\end{Verbatim}
\end{quote}


\section{stardic}
\label{index:stardic}\begin{quote}

stardict离线词典安装:将下载的字典解压得到dic文件夹

\begin{Verbatim}[commandchars=\\\{\}]
chmod a+rx \PYGZhy{}R dic   \PYG{c}{\PYGZsh{}修改权限,否则程序无法读取字典}
sudo mv dic /usr/share/stardict/
\end{Verbatim}
\end{quote}


\section{Terminal}
\label{index:terminal}\begin{quote}

右键中添加open in terminal选项

\begin{Verbatim}[commandchars=\\\{\}]
sudo apt\PYGZhy{}get install nautilus\PYGZhy{}open\PYGZhy{}terminal
\end{Verbatim}
\end{quote}


\chapter{环境变量}
\label{index:id3}\begin{quote}
\begin{enumerate}
\item {} 
/etc/profile, /etc/bashrc, .bash\_profile, .bashrc

\end{enumerate}
\begin{quote}

\emph{/etc/profile}
\begin{quote}

其设定的变量可以作用于所有用户。此文件为系统的每个用户设置环境信息,当用户第一次登录时,该文件被执行. 并从/etc/profile.d目录的配置文件中搜集shell的设置。
\end{quote}

\emph{/etc/bashrc}
\begin{quote}

为每一个运行bash shell的用户执行此文件.当bash shell被打开时,该文件被读取.Ubuntu没有此文件,与之对应的是/ect/bash.bashrc它也是全局(公有)的 bash执行时,不管是何种方式,都会读取此文件。
\end{quote}

\emph{\textasciitilde{}/.bash\_profile}
\begin{quote}

是交互式、login 方式进入 bash 运行的 每个用户都可使用该文件输入专用于自己使用的shell信息,当用户登录时,该文件仅仅执行一次! 默认情况下, 他设置一些环境变量,执行用户的.bashrc文件. Unbutu默认没有此文件,可新建。只有bash是以login形式执行时,才会读取此文件。通常该配置文件还会配置成去读取\textasciitilde{}/.bashrc。
\end{quote}

\emph{\textasciitilde{}/.bashrc}
\begin{quote}

是交互式 non-login 方式进入 bash 运行的。设定的变量(局部)只能继承/etc/profile中的变量,他们是''父子''关系. 该文件包含专用于你的bash shell的bash信息,当登录时以及每次打开新的shell时,该文件被读取. 当bash是以non-login形式执行时,读取此文件。若是以login形式执行,则不会读取此文件
\end{quote}

\emph{\textasciitilde{}/.profile}
\begin{quote}

若bash是以login方式执行时,读取\textasciitilde{}/.bash\_profile,若它不存在.bash\_login,若前两者不存在,读取\textasciitilde{}/.profile。 另外,图形模式登录时,此文件将被读取,即使存在\textasciitilde{}/.bash\_profile和\textasciitilde{}/.bash\_login。
\end{quote}

\emph{\textasciitilde{}/.bash\_login}
\begin{quote}

若bash是以login方式执行时,读取\textasciitilde{}/.bash\_profile,若它不存在.bash\_login,若前两者不存在,读取\textasciitilde{}/.profile。
\end{quote}
\begin{description}
\item[{\emph{\textasciitilde{}/.bash\_logout}}] \leavevmode
当每次退出系统(退出bash shell)时,执行该文件.注销时,且是longin形式,此文件才会读取。也就是说,在文本模式注销时,此文件会被读取,图形模式注不会被读取。

\end{description}

通常二者设置大致相同,所以通常前者会调用后者。

下面是几个例子:
\begin{enumerate}
\item {} 
图形模式登录时,顺序读取:/etc/profile和\textasciitilde{}/.profile

\item {} 
图形模式登录后,打开终端时,顺序读取:/etc/bash.bashrc和\textasciitilde{}/.bashrc

\item {} 
文本模式登录时,顺序读取:/etc/bash.bashrc,/etc/profile和\textasciitilde{}/.bash\_profile

\item {} 
从其它用户su到该用户,则分两种情况:
\begin{itemize}
\item {} 
如果带-l参数(或-参数,--login参数),如:su -l username,则bash是lonin的,它将顺序读取以下配置文件:/etc/bash.bashrc,/etc/profile和\textasciitilde{}/.   bash\_profile。

\item {} 
如果没有带-l参数,则bash是non-login的,它将顺序读取:/etc/bash.bashrc和\textasciitilde{}/.bashrc

\end{itemize}

\item {} 
注销时,或退出su登录的用户,如果是longin方式,那么bash会读取:\textasciitilde{}/.bash\_logout

\item {} 
执行自定义的shell文件时,若使用“bash -l a.sh”的方式,则bash会读取行:/etc/profile和\textasciitilde{}/.bash\_profile,若使用其它方式,如:bash a.sh, ./a.sh,sh a.sh(这个不属于bash shell),则不会读取上面的任何文件。

\item {} 
上面的例子凡是读取到\textasciitilde{}/.bash\_profile的,若该文件不存在,则读取\textasciitilde{}/.bash\_login,若前两者不存在,读取\textasciitilde{}/.profile。

\end{enumerate}
\end{quote}
\begin{enumerate}
\setcounter{enumi}{1}
\item {} 
修改变量:

\end{enumerate}
\begin{quote}
\begin{enumerate}
\item {} 
在终端修改,但只对当前终端有效:

\end{enumerate}
\begin{quote}

\begin{Verbatim}[commandchars=\\\{\}]
export PATH=\PYGZdl{}PATH:/etc/apache/bin
\end{Verbatim}
\end{quote}
\begin{enumerate}
\setcounter{enumi}{1}
\item {} 
在 \emph{/etc/profile} 中添加(注意:等号两边不能由空格),这种方法最好,除非手动强制修改,否则不会被改变

\item {} 
在 \emph{\textasciitilde{}/.bashrc} 中修改,只对当前用户管用。

\item {} 
在 \emph{\textasciitilde{}/.bashr\_profile} 中修改,也只对当前用户有效。比如在root权限下操作,则对root用户有效。

\end{enumerate}
\end{quote}
\begin{quote}\begin{description}
\item[{注意}] \leavevmode
要使修改的变量生效,必须重新登录,或编辑结束后对所编辑的文件使用source命令:

\begin{Verbatim}[commandchars=\\\{\}]
source filename
\end{Verbatim}

\end{description}\end{quote}
\end{quote}


\chapter{双系统时间差}
\label{index:id4}\begin{quote}
\begin{enumerate}
\item {} 
Windows与Linux(Ubuntu)双系统时间不一致 (相差8小时) 的解决方法:

\end{enumerate}
\begin{itemize}
\item {} 
Ubuntu下修改:(推荐)

Ubuntu中不使用UTC时间,而启用本地时间,需要修改 /etc/default/rcS ,修改动作如下:
\# 注释掉原来的设定:UTC=yes
\# 变更为下面的内容...
UTC=no

\item {} 
windows下修改:

在 注册表项:HKEY\_LOCAL\_MACHINE\textbackslash{}SYSTEM\textbackslash{}CurrentControlSet\textbackslash{}Control\textbackslash{}TimeZoneInformation\textbackslash{}
下中添加一项数据类型为REG\_DWORD,名称为RealTimeIsUniversal,值设为1 的键值。
或者直接使用下如下批处理进行修改:

\begin{Verbatim}[commandchars=\\\{\}]
@echo off
color 0a
Reg add HKLM\PYGZbs{}SYSTEM\PYGZbs{}CurrentControlSet\PYGZbs{}Control\PYGZbs{}TimeZoneInformation /v RealTimeIsUniversal /t REG\PYGZus{}DWORD /d 1
echo.
echo 已让Windows识别存贮在主板CMOS内的时间为格林威治标准时间(GMT),即系统根据CMOS时间和设置的时区来确定当前系统的时间。
echo.
pause
\end{Verbatim}

\end{itemize}
\end{quote}


\chapter{输入密码后出现闪屏又回到登录界面}
\label{index:id5}\begin{quote}
\begin{description}
\item[{\emph{原因}}] \leavevmode
手贱在自己账户下 命令行 \emph{sudo startX}

\item[{\emph{结果}}] \leavevmode
只能访问 \emph{guest session}

\item[{\emph{解决}}] \leavevmode
由于 \emph{sudo startX} 操作后,将 \emph{\textasciitilde{}/.Xauthority} 文件所有者改为了root,
因此只要tty命令行下登录用户,\emph{sudo chown acount:acount \textasciitilde{}/.Xauthority} 。
ctrl + alt + F7

\end{description}
\end{quote}


\chapter{Linux制作U盘启动盘}
\label{index:linuxu}\begin{itemize}
\item {} 
cat

使用方法: cat  镜像目录 \textgreater{} 设备(可以使用fdisk -l 查看)。

\begin{Verbatim}[commandchars=\\\{\}]
cat ubuntu\PYGZhy{}13.10\PYGZhy{}desktop\PYGZhy{}i386.iso \PYGZgt{}/dev/sdb1
\end{Verbatim}
\begin{quote}\begin{description}
\item[{注意}] \leavevmode
后面重定向的是设备而不是设备挂载后的目录,还有就是,使用 root 权限运行。

\end{description}\end{quote}

\item {} 
dd

使用方法: dd {[}选项{]}  (最好自己看一下 dd 的 manpage)。
\begin{quote}\begin{description}
\item[{功能}] \leavevmode
把指定的输入文件拷贝到指定的输出文件中,并且在拷贝过程中可以进行格式转换。
可以用该命令实现DOS下的diskcopy命令的作用。先用dd命令把软盘上的数据写成硬盘的一个寄存文件,
再把这个寄存文件写入第二张软盘上,完成diskcopy的功能。
需要注意的是,应该将硬盘上的寄存文件用rm命令删除掉。系统默认使用标准输入文件和标准输出文件。

\end{description}\end{quote}

那么,我们可以看到 dd 主要的功能就是拷贝文件,并且在拷贝时可以格式转换,所以 dd 也是平时使用比较多的进行制作 U 盘(或者 cdrom)的工具。我看道网上说甚至可以使用 dd 将 CD 上的文件拷贝到本地并放在一个镜像中(我们可以使用这个来实现将 CD/DVD上的系统拷贝到本地作为镜像文件)。

\begin{Verbatim}[commandchars=\\\{\}]
\PYG{c}{\PYGZsh{}使用 dd 制作 U 盘启动盘:}
dd \PYG{k}{if}\PYG{o}{=}ubuntu\PYGZhy{}13.10\PYGZhy{}desktop\PYGZhy{}i386.iso  \PYG{n+nv}{of}\PYG{o}{=}/dev/sdb1 \PYG{n+nv}{bs}\PYG{o}{=}4M
\end{Verbatim}

我们可以看到,使用 dd 只是和使用 cat 带入参数方法不一样,镜像目录使用 if= 来指定,U盘设备的地址使用 of= 来指定,最后的 bs=4M 上面也可以看到是指定一次写入字节数

\begin{Verbatim}[commandchars=\\\{\}]
\PYG{c}{\PYGZsh{}使用 dd 制作 CD/DVD 系统盘:}
dd \PYG{k}{if}\PYG{o}{=}ubuntu\PYGZhy{}13.10\PYGZhy{}desktop\PYGZhy{}i386.iso  \PYG{n+nv}{of}\PYG{o}{=}/dev/cdrom \PYG{n+nv}{bs}\PYG{o}{=}4M

\PYG{c}{\PYGZsh{}使用 dd 制作镜像:}
dd \PYG{k}{if}\PYG{o}{=}/dev/cdrom \PYG{n+nv}{of}\PYG{o}{=}/home/username/cd.iso  \PYG{n+nv}{bs}\PYG{o}{=}4M
\end{Verbatim}
\begin{quote}\begin{description}
\item[{注}] \leavevmode\begin{itemize}
\item {} 
bs 带的参数是一次写入字节数,可以自己替换更高的数来得到更高的速度.但是真正使用时,速度的上限还是限于硬件的读写速度(比如说,USB2.0 的U盘,你带入参数 bs=16M,但是真正得到的读写速度仍然是3M~4M的样子)

\item {} 
使用 root 权限运行

\end{itemize}

\end{description}\end{quote}

\end{itemize}


\chapter{乱码}
\label{index:id6}\begin{quote}
\begin{enumerate}
\item {} 
Rhythmbox

\end{enumerate}
\begin{quote}

编辑/etc/profile:

添加并保存:

\begin{Verbatim}[commandchars=\\\{\}]
export GST\PYGZus{}ID3\PYGZus{}TAG\PYGZus{}ENCODING=GBK:UTF\PYGZhy{}8:GB18030
export GST\PYGZus{}ID3V2\PYGZus{}TAG\PYGZus{}ENCODING=GBK:UTF\PYGZhy{}8:GB18030
\end{Verbatim}
\end{quote}
\begin{enumerate}
\setcounter{enumi}{1}
\item {} 
gedit

\end{enumerate}
\begin{quote}

gedit打开GB18030/GBK/GB2312 等类型的中文编码文本文件时,将会出现乱码。
\begin{description}
\item[{\emph{原因}}] \leavevmode
gedit 使用一个编码匹配列表,只有在这个列表中的编码才会进行匹配,不在这个列表中的编码将显示为乱码。

\item[{\emph{解决}}] \leavevmode
要做的就是将GB18030 加入这个匹配列表

\begin{Verbatim}[commandchars=\\\{\}]
gsettings \PYG{n+nb}{set }org.gnome.gedit.preferences.encodings auto\PYGZhy{}detected \PYG{l+s+s2}{\PYGZdq{}[\PYGZsq{}UTF\PYGZhy{}8\PYGZsq{},\PYGZsq{}GB18030\PYGZsq{},\PYGZsq{}GB2312\PYGZsq{},\PYGZsq{}GBK\PYGZsq{},\PYGZsq{}BIG5\PYGZsq{},\PYGZsq{}CURRENT\PYGZsq{},\PYGZsq{}UTF\PYGZhy{}16\PYGZsq{}]\PYGZdq{}}
\end{Verbatim}

若出现如下错误:

\begin{Verbatim}[commandchars=\\\{\}]
** (gedit:3525): WARNING **: Could not load Gedit repository: Typelib file for namespace \PYGZsq{}GtkSource\PYGZsq{}, version \PYGZsq{}3.0\PYGZsq{} not found
\end{Verbatim}

则:

\begin{Verbatim}[commandchars=\\\{\}]
sudo apt\PYGZhy{}get install gir1.2\PYGZhy{}gtksource\PYGZhy{}3.0
\end{Verbatim}

\end{description}
\end{quote}
\end{quote}



\renewcommand{\indexname}{Index}
\printindex
\end{document}
